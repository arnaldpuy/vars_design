\PassOptionsToPackage{unicode=true}{hyperref} % options for packages loaded elsewhere
\PassOptionsToPackage{hyphens}{url}
%
\documentclass[11pt,]{article}
\usepackage{lmodern}
\usepackage{amssymb,amsmath}
\usepackage{ifxetex,ifluatex}
\usepackage{fixltx2e} % provides \textsubscript
\ifnum 0\ifxetex 1\fi\ifluatex 1\fi=0 % if pdftex
  \usepackage[T1]{fontenc}
  \usepackage[utf8]{inputenc}
  \usepackage{textcomp} % provides euro and other symbols
\else % if luatex or xelatex
  \usepackage{unicode-math}
  \defaultfontfeatures{Ligatures=TeX,Scale=MatchLowercase}
\fi
% use upquote if available, for straight quotes in verbatim environments
\IfFileExists{upquote.sty}{\usepackage{upquote}}{}
% use microtype if available
\IfFileExists{microtype.sty}{%
\usepackage[]{microtype}
\UseMicrotypeSet[protrusion]{basicmath} % disable protrusion for tt fonts
}{}
\IfFileExists{parskip.sty}{%
\usepackage{parskip}
}{% else
\setlength{\parindent}{0pt}
\setlength{\parskip}{6pt plus 2pt minus 1pt}
}
\usepackage{hyperref}
\hypersetup{
            pdftitle={The Revenge Paper},
            pdfauthor={Arnald Puy, Samuele Lo Piano, Andrea Saltelli},
            pdfborder={0 0 0},
            breaklinks=true}
\urlstyle{same}  % don't use monospace font for urls
\usepackage[margin=1in]{geometry}
\usepackage{color}
\usepackage{fancyvrb}
\newcommand{\VerbBar}{|}
\newcommand{\VERB}{\Verb[commandchars=\\\{\}]}
\DefineVerbatimEnvironment{Highlighting}{Verbatim}{commandchars=\\\{\}}
% Add ',fontsize=\small' for more characters per line
\usepackage{framed}
\definecolor{shadecolor}{RGB}{248,248,248}
\newenvironment{Shaded}{\begin{snugshade}}{\end{snugshade}}
\newcommand{\AlertTok}[1]{\textcolor[rgb]{0.94,0.16,0.16}{#1}}
\newcommand{\AnnotationTok}[1]{\textcolor[rgb]{0.56,0.35,0.01}{\textbf{\textit{#1}}}}
\newcommand{\AttributeTok}[1]{\textcolor[rgb]{0.77,0.63,0.00}{#1}}
\newcommand{\BaseNTok}[1]{\textcolor[rgb]{0.00,0.00,0.81}{#1}}
\newcommand{\BuiltInTok}[1]{#1}
\newcommand{\CharTok}[1]{\textcolor[rgb]{0.31,0.60,0.02}{#1}}
\newcommand{\CommentTok}[1]{\textcolor[rgb]{0.56,0.35,0.01}{\textit{#1}}}
\newcommand{\CommentVarTok}[1]{\textcolor[rgb]{0.56,0.35,0.01}{\textbf{\textit{#1}}}}
\newcommand{\ConstantTok}[1]{\textcolor[rgb]{0.00,0.00,0.00}{#1}}
\newcommand{\ControlFlowTok}[1]{\textcolor[rgb]{0.13,0.29,0.53}{\textbf{#1}}}
\newcommand{\DataTypeTok}[1]{\textcolor[rgb]{0.13,0.29,0.53}{#1}}
\newcommand{\DecValTok}[1]{\textcolor[rgb]{0.00,0.00,0.81}{#1}}
\newcommand{\DocumentationTok}[1]{\textcolor[rgb]{0.56,0.35,0.01}{\textbf{\textit{#1}}}}
\newcommand{\ErrorTok}[1]{\textcolor[rgb]{0.64,0.00,0.00}{\textbf{#1}}}
\newcommand{\ExtensionTok}[1]{#1}
\newcommand{\FloatTok}[1]{\textcolor[rgb]{0.00,0.00,0.81}{#1}}
\newcommand{\FunctionTok}[1]{\textcolor[rgb]{0.00,0.00,0.00}{#1}}
\newcommand{\ImportTok}[1]{#1}
\newcommand{\InformationTok}[1]{\textcolor[rgb]{0.56,0.35,0.01}{\textbf{\textit{#1}}}}
\newcommand{\KeywordTok}[1]{\textcolor[rgb]{0.13,0.29,0.53}{\textbf{#1}}}
\newcommand{\NormalTok}[1]{#1}
\newcommand{\OperatorTok}[1]{\textcolor[rgb]{0.81,0.36,0.00}{\textbf{#1}}}
\newcommand{\OtherTok}[1]{\textcolor[rgb]{0.56,0.35,0.01}{#1}}
\newcommand{\PreprocessorTok}[1]{\textcolor[rgb]{0.56,0.35,0.01}{\textit{#1}}}
\newcommand{\RegionMarkerTok}[1]{#1}
\newcommand{\SpecialCharTok}[1]{\textcolor[rgb]{0.00,0.00,0.00}{#1}}
\newcommand{\SpecialStringTok}[1]{\textcolor[rgb]{0.31,0.60,0.02}{#1}}
\newcommand{\StringTok}[1]{\textcolor[rgb]{0.31,0.60,0.02}{#1}}
\newcommand{\VariableTok}[1]{\textcolor[rgb]{0.00,0.00,0.00}{#1}}
\newcommand{\VerbatimStringTok}[1]{\textcolor[rgb]{0.31,0.60,0.02}{#1}}
\newcommand{\WarningTok}[1]{\textcolor[rgb]{0.56,0.35,0.01}{\textbf{\textit{#1}}}}
\usepackage{graphicx,grffile}
\makeatletter
\def\maxwidth{\ifdim\Gin@nat@width>\linewidth\linewidth\else\Gin@nat@width\fi}
\def\maxheight{\ifdim\Gin@nat@height>\textheight\textheight\else\Gin@nat@height\fi}
\makeatother
% Scale images if necessary, so that they will not overflow the page
% margins by default, and it is still possible to overwrite the defaults
% using explicit options in \includegraphics[width, height, ...]{}
\setkeys{Gin}{width=\maxwidth,height=\maxheight,keepaspectratio}
\setlength{\emergencystretch}{3em}  % prevent overfull lines
\providecommand{\tightlist}{%
  \setlength{\itemsep}{0pt}\setlength{\parskip}{0pt}}
\setcounter{secnumdepth}{5}
% Redefines (sub)paragraphs to behave more like sections
\ifx\paragraph\undefined\else
\let\oldparagraph\paragraph
\renewcommand{\paragraph}[1]{\oldparagraph{#1}\mbox{}}
\fi
\ifx\subparagraph\undefined\else
\let\oldsubparagraph\subparagraph
\renewcommand{\subparagraph}[1]{\oldsubparagraph{#1}\mbox{}}
\fi

% set default figure placement to htbp
\makeatletter
\def\fps@figure{htbp}
\makeatother

\usepackage[font=footnotesize]{caption}
\usepackage{dirtytalk}
\usepackage{booktabs}
\usepackage{tabulary}
\usepackage{enumitem}
\usepackage{lmodern}
\usepackage[T1]{fontenc}

\title{The Revenge Paper}
\author{Arnald Puy, Samuele Lo Piano, Andrea Saltelli}
\date{}

\begin{document}
\maketitle

{
\setcounter{tocdepth}{2}
\tableofcontents
}
\newpage

\hypertarget{preliminary-functions}{%
\section{Preliminary functions}\label{preliminary-functions}}

\begin{Shaded}
\begin{Highlighting}[]
\CommentTok{# PRELIMINARY FUNCTIONS -------------------------------------------------------}

\CommentTok{# Function to read in all required packages in one go:}
\NormalTok{loadPackages <-}\StringTok{ }\ControlFlowTok{function}\NormalTok{(x) \{}
  \ControlFlowTok{for}\NormalTok{(i }\ControlFlowTok{in}\NormalTok{ x) \{}
    \ControlFlowTok{if}\NormalTok{(}\OperatorTok{!}\KeywordTok{require}\NormalTok{(i, }\DataTypeTok{character.only =} \OtherTok{TRUE}\NormalTok{)) \{}
      \KeywordTok{install.packages}\NormalTok{(i, }\DataTypeTok{dependencies =} \OtherTok{TRUE}\NormalTok{)}
      \KeywordTok{library}\NormalTok{(i, }\DataTypeTok{character.only =} \OtherTok{TRUE}\NormalTok{)}
\NormalTok{    \}}
\NormalTok{  \}}
\NormalTok{\}}

\CommentTok{# Load the packages}
\KeywordTok{loadPackages}\NormalTok{(}\KeywordTok{c}\NormalTok{(}\StringTok{"tidyverse"}\NormalTok{))}

\CommentTok{# Create custom theme}
\NormalTok{theme_AP <-}\StringTok{ }\ControlFlowTok{function}\NormalTok{() \{}
  \KeywordTok{theme_bw}\NormalTok{() }\OperatorTok{+}
\StringTok{    }\KeywordTok{theme}\NormalTok{(}\DataTypeTok{panel.grid.major =} \KeywordTok{element_blank}\NormalTok{(),}
          \DataTypeTok{panel.grid.minor =} \KeywordTok{element_blank}\NormalTok{(),}
          \DataTypeTok{legend.background =} \KeywordTok{element_rect}\NormalTok{(}\DataTypeTok{fill =} \StringTok{"transparent"}\NormalTok{,}
                                           \DataTypeTok{color =} \OtherTok{NA}\NormalTok{),}
          \DataTypeTok{legend.key =} \KeywordTok{element_rect}\NormalTok{(}\DataTypeTok{fill =} \StringTok{"transparent"}\NormalTok{,}
                                    \DataTypeTok{color =} \OtherTok{NA}\NormalTok{))}
\NormalTok{\}}

\CommentTok{# Set checkpoint}

\KeywordTok{dir.create}\NormalTok{(}\StringTok{".checkpoint"}\NormalTok{)}
\KeywordTok{library}\NormalTok{(}\StringTok{"checkpoint"}\NormalTok{)}

\KeywordTok{checkpoint}\NormalTok{(}\StringTok{"2020-03-09"}\NormalTok{, }
           \DataTypeTok{R.version =}\StringTok{"3.6.1"}\NormalTok{, }
           \DataTypeTok{checkpointLocation =} \KeywordTok{getwd}\NormalTok{())}
\end{Highlighting}
\end{Shaded}

\newpage

Here we define the code to produce STAR-VARS matrices using Sobol'
quasi-random number sequences:

\begin{Shaded}
\begin{Highlighting}[]
\CommentTok{# FUNCTION TO CREATE STAR-VARS ------------------------------------------------}

\NormalTok{star_vars <-}\StringTok{ }\ControlFlowTok{function}\NormalTok{(N, params, h) \{}
\NormalTok{  out <-}\StringTok{ }\NormalTok{center <-}\StringTok{ }\NormalTok{sections <-}\StringTok{ }\NormalTok{A <-}\StringTok{ }\NormalTok{B <-}\StringTok{ }\NormalTok{AB <-}\StringTok{ }\NormalTok{X <-}\StringTok{ }\NormalTok{out <-}\StringTok{ }\KeywordTok{list}\NormalTok{()}
\NormalTok{  mat <-}\StringTok{ }\NormalTok{randtoolbox}\OperatorTok{::}\KeywordTok{sobol}\NormalTok{(}\DataTypeTok{n =}\NormalTok{ N, }\DataTypeTok{dim =} \KeywordTok{length}\NormalTok{(params))}
  \ControlFlowTok{for}\NormalTok{(i }\ControlFlowTok{in} \DecValTok{1}\OperatorTok{:}\KeywordTok{nrow}\NormalTok{(mat)) \{}
\NormalTok{    center[[i]] <-}\StringTok{ }\NormalTok{mat[i, ]}
\NormalTok{    sections[[i]] <-}\StringTok{ }\KeywordTok{sapply}\NormalTok{(center[[i]], }\ControlFlowTok{function}\NormalTok{(x) \{}
\NormalTok{      all <-}\StringTok{ }\KeywordTok{seq}\NormalTok{(x }\OperatorTok\StringTok{ }\NormalTok{h, }\DecValTok{1}\NormalTok{, h)}
\NormalTok{      non.zeros <-}\StringTok{ }\NormalTok{all[all}\OperatorTok{!=}\StringTok{ }\DecValTok{0}\NormalTok{] }\CommentTok{# Remove zeroes}
\NormalTok{      \})}
\NormalTok{    B[[i]] <-}\StringTok{ }\KeywordTok{sapply}\NormalTok{(}\DecValTok{1}\OperatorTok{:}\KeywordTok{ncol}\NormalTok{(mat), }\ControlFlowTok{function}\NormalTok{(x)}
\NormalTok{      sections[[i]][, x][}\OperatorTok{!}\NormalTok{sections[[i]][, x] }\OperatorTok\StringTok{ }\NormalTok{center[[i]][x]])}
\NormalTok{    A[[i]] <-}\StringTok{ }\KeywordTok{matrix}\NormalTok{(center[[i]], }\DataTypeTok{nrow =} \KeywordTok{nrow}\NormalTok{(B[[i]]), }
                     \DataTypeTok{ncol =} \KeywordTok{length}\NormalTok{(center[[i]]), }\DataTypeTok{byrow =} \OtherTok{TRUE}\NormalTok{)}
\NormalTok{    X[[i]] <-}\StringTok{ }\KeywordTok{rbind}\NormalTok{(A[[i]], B[[i]])}
    \ControlFlowTok{for}\NormalTok{(j }\ControlFlowTok{in} \DecValTok{1}\OperatorTok{:}\KeywordTok{ncol}\NormalTok{(A[[i]])) \{}
\NormalTok{      AB[[i]] <-}\StringTok{ }\NormalTok{A[[i]]}
\NormalTok{      AB[[i]][, j] <-}\StringTok{ }\NormalTok{B[[i]][, j]}
\NormalTok{      X[[i]] <-}\StringTok{ }\KeywordTok{rbind}\NormalTok{(X[[i]], AB[[i]])}
\NormalTok{    \}}
\NormalTok{    AB[[i]] <-}\StringTok{ }\NormalTok{X[[i]][(}\DecValTok{2} \OperatorTok{*}\StringTok{ }\KeywordTok{nrow}\NormalTok{(B[[i]]) }\OperatorTok{+}\StringTok{ }\DecValTok{1}\NormalTok{)}\OperatorTok{:}\KeywordTok{nrow}\NormalTok{(X[[i]]), ]}
\NormalTok{    out[[i]] <-}\StringTok{ }\KeywordTok{rbind}\NormalTok{(}\KeywordTok{unname}\NormalTok{(center[[i]]), AB[[i]])}
\NormalTok{  \}}
  \KeywordTok{return}\NormalTok{(}\KeywordTok{do.call}\NormalTok{(rbind, out))}
\NormalTok{\}}

\CommentTok{# Function to split matrices in n chunks}
\NormalTok{CutBySize <-}\StringTok{ }\ControlFlowTok{function}\NormalTok{(m, block.size, }\DataTypeTok{nb =} \KeywordTok{ceiling}\NormalTok{(m }\OperatorTok{/}\StringTok{ }\NormalTok{block.size)) \{}
\NormalTok{  int <-}\StringTok{ }\NormalTok{m }\OperatorTok{/}\StringTok{ }\NormalTok{nb}
\NormalTok{  upper <-}\StringTok{ }\KeywordTok{round}\NormalTok{(}\DecValTok{1}\OperatorTok{:}\NormalTok{nb }\OperatorTok{*}\StringTok{ }\NormalTok{int)}
\NormalTok{  lower <-}\StringTok{ }\KeywordTok{c}\NormalTok{(}\DecValTok{1}\NormalTok{, upper[}\OperatorTok{-}\NormalTok{nb] }\OperatorTok{+}\StringTok{ }\DecValTok{1}\NormalTok{)}
\NormalTok{  size <-}\StringTok{ }\KeywordTok{c}\NormalTok{(upper[}\DecValTok{1}\NormalTok{], }\KeywordTok{diff}\NormalTok{(upper))}
  \KeywordTok{cbind}\NormalTok{(lower, upper, size)}
\NormalTok{\}}
\end{Highlighting}
\end{Shaded}

Now we create a STAR-VARS sample matrix with two star centers, \(h=0.1\)
and three model inputs.

\begin{Shaded}
\begin{Highlighting}[]
\CommentTok{# DEFINE THE SETTINGS FOR A STAR-VARS SAMPLE MATRIX ---------------------------}

\NormalTok{N <-}\StringTok{ }\DecValTok{2}
\NormalTok{params <-}\StringTok{ }\KeywordTok{paste}\NormalTok{(}\StringTok{"X"}\NormalTok{, }\DecValTok{1}\OperatorTok{:}\DecValTok{3}\NormalTok{, }\DataTypeTok{sep =} \StringTok{""}\NormalTok{)}
\NormalTok{h <-}\StringTok{ }\FloatTok{0.1}
\end{Highlighting}
\end{Shaded}

\begin{Shaded}
\begin{Highlighting}[]
\CommentTok{# CREATE STAR-VARS ------------------------------------------------------------}

\NormalTok{mat <-}\StringTok{ }\KeywordTok{star_vars}\NormalTok{(}\DataTypeTok{N =}\NormalTok{ N, }\DataTypeTok{params =}\NormalTok{ params, }\DataTypeTok{h =}\NormalTok{ h)}

\KeywordTok{print}\NormalTok{(mat)}
\end{Highlighting}
\end{Shaded}

\begin{verbatim}
##       [,1] [,2] [,3]
##  [1,] 0.50 0.50 0.50
##  [2,] 0.10 0.50 0.50
##  [3,] 0.20 0.50 0.50
##  [4,] 0.30 0.50 0.50
##  [5,] 0.40 0.50 0.50
##  [6,] 0.60 0.50 0.50
##  [7,] 0.70 0.50 0.50
##  [8,] 0.80 0.50 0.50
##  [9,] 0.90 0.50 0.50
## [10,] 1.00 0.50 0.50
## [11,] 0.50 0.10 0.50
## [12,] 0.50 0.20 0.50
## [13,] 0.50 0.30 0.50
## [14,] 0.50 0.40 0.50
## [15,] 0.50 0.60 0.50
## [16,] 0.50 0.70 0.50
## [17,] 0.50 0.80 0.50
## [18,] 0.50 0.90 0.50
## [19,] 0.50 1.00 0.50
## [20,] 0.50 0.50 0.10
## [21,] 0.50 0.50 0.20
## [22,] 0.50 0.50 0.30
## [23,] 0.50 0.50 0.40
## [24,] 0.50 0.50 0.60
## [25,] 0.50 0.50 0.70
## [26,] 0.50 0.50 0.80
## [27,] 0.50 0.50 0.90
## [28,] 0.50 0.50 1.00
## [29,] 0.75 0.25 0.75
## [30,] 0.05 0.25 0.75
## [31,] 0.15 0.25 0.75
## [32,] 0.25 0.25 0.75
## [33,] 0.35 0.25 0.75
## [34,] 0.45 0.25 0.75
## [35,] 0.55 0.25 0.75
## [36,] 0.65 0.25 0.75
## [37,] 0.85 0.25 0.75
## [38,] 0.95 0.25 0.75
## [39,] 0.75 0.05 0.75
## [40,] 0.75 0.15 0.75
## [41,] 0.75 0.35 0.75
## [42,] 0.75 0.45 0.75
## [43,] 0.75 0.55 0.75
## [44,] 0.75 0.65 0.75
## [45,] 0.75 0.75 0.75
## [46,] 0.75 0.85 0.75
## [47,] 0.75 0.95 0.75
## [48,] 0.75 0.25 0.05
## [49,] 0.75 0.25 0.15
## [50,] 0.75 0.25 0.25
## [51,] 0.75 0.25 0.35
## [52,] 0.75 0.25 0.45
## [53,] 0.75 0.25 0.55
## [54,] 0.75 0.25 0.65
## [55,] 0.75 0.25 0.85
## [56,] 0.75 0.25 0.95
\end{verbatim}

Note that the first star center is in row 1; the second star center is
in line 29. All stars thus have \(k((1/h) - 1)+1\) points (including the
center), 28 rows per star in this specific case.

Now we compute the model output with the Ishigami function:

\begin{Shaded}
\begin{Highlighting}[]
\CommentTok{# MODEL OUTPUT ----------------------------------------------------------------}

\NormalTok{Y <-}\StringTok{ }\NormalTok{sensobol}\OperatorTok{::}\KeywordTok{ishigami_Fun}\NormalTok{(mat)}
\end{Highlighting}
\end{Shaded}

Now we reorganize the data:

\begin{Shaded}
\begin{Highlighting}[]
\CommentTok{# REARRANGE DATA --------------------------------------------------------------}

\NormalTok{n.cross.points <-}\StringTok{ }\KeywordTok{length}\NormalTok{(params) }\OperatorTok{*}\StringTok{ }\NormalTok{((}\DecValTok{1} \OperatorTok{/}\StringTok{ }\NormalTok{h) }\OperatorTok{-}\StringTok{ }\DecValTok{1}\NormalTok{) }\OperatorTok{+}\StringTok{ }\DecValTok{1}
\NormalTok{index.centers <-}\StringTok{ }\KeywordTok{seq}\NormalTok{(}\DecValTok{1}\NormalTok{, }\KeywordTok{nrow}\NormalTok{(mat), n.cross.points)}
\NormalTok{mat.nocenters <-}\StringTok{ }\KeywordTok{matrix}\NormalTok{(Y[}\OperatorTok{-}\NormalTok{index.centers], }\DataTypeTok{ncol =}\NormalTok{ N)}
\NormalTok{mat.centers <-}\StringTok{ }\KeywordTok{matrix}\NormalTok{(Y[index.centers], }
                      \DataTypeTok{nrow =} \KeywordTok{nrow}\NormalTok{(mat.nocenters) }\OperatorTok{+}\StringTok{ }\KeywordTok{length}\NormalTok{(params),}
                      \DataTypeTok{ncol =}\NormalTok{ N, }
                      \DataTypeTok{byrow =} \OtherTok{TRUE}\NormalTok{)}
\NormalTok{location.centers <-}\StringTok{ }\KeywordTok{seq}\NormalTok{(}\DecValTok{1}\NormalTok{, }\KeywordTok{nrow}\NormalTok{(mat.nocenters), }\DecValTok{1} \OperatorTok{/}\StringTok{ }\NormalTok{h)}
\NormalTok{mat.centers[}\OperatorTok{-}\NormalTok{location.centers, ] <-}\StringTok{ }\NormalTok{mat.nocenters}

\NormalTok{indices <-}\StringTok{ }\KeywordTok{CutBySize}\NormalTok{(}\KeywordTok{nrow}\NormalTok{(mat.centers), }\DataTypeTok{nb =} \KeywordTok{length}\NormalTok{(params))}
\NormalTok{out <-}\StringTok{ }\NormalTok{da <-}\StringTok{ }\KeywordTok{list}\NormalTok{()}
\ControlFlowTok{for}\NormalTok{(i }\ControlFlowTok{in} \DecValTok{1}\OperatorTok{:}\KeywordTok{nrow}\NormalTok{(indices)) \{}
\NormalTok{  out[[i]] <-}\StringTok{ }\NormalTok{mat.centers[indices[i, }\StringTok{"lower"}\NormalTok{]}\OperatorTok{:}\NormalTok{indices[i, }\StringTok{"upper"}\NormalTok{], ]}
\NormalTok{\}}

\KeywordTok{print}\NormalTok{(out)}
\end{Highlighting}
\end{Shaded}

\begin{verbatim}
## [[1]]
##                [,1]       [,2]
##  [1,]  0.000000e+00  9.0880682
##  [2,] -5.877853e-01 -0.1903335
##  [3,] -9.510565e-01 -3.7343676
##  [4,] -9.510565e-01 -5.0880682
##  [5,] -5.877853e-01 -3.7343676
##  [6,]  5.877853e-01 -0.1903335
##  [7,]  9.510565e-01  4.1903335
##  [8,]  9.510565e-01  7.7343676
##  [9,]  5.877853e-01  7.7343676
## [10,]  1.224647e-16  4.1903335
## 
## [[2]]
##               [,1]     [,2]
##  [1,] 0.000000e+00 9.088068
##  [2,] 6.909830e-01 7.279051
##  [3,] 1.809017e+00 8.397085
##  [4,] 1.809017e+00 8.397085
##  [5,] 6.909830e-01 7.279051
##  [6,] 6.909830e-01 7.279051
##  [7,] 1.809017e+00 8.397085
##  [8,] 1.809017e+00 9.088068
##  [9,] 6.909830e-01 8.397085
## [10,] 2.999520e-32 7.279051
## 
## [[3]]
##       [,1]      [,2]
##  [1,]    0  9.088068
##  [2,]    0 66.910105
##  [3,]    0 26.387923
##  [4,]    0  9.088068
##  [5,]    0  3.789014
##  [6,]    0  3.009741
##  [7,]    0  3.009741
##  [8,]    0  3.789014
##  [9,]    0 26.387923
## [10,]    0 66.910105
\end{verbatim}

The list of matrices above includes three slots for three model inputs.
Each slot has two columns, each column being a star. The first row of
each slot includes the star center.

According to Razavi and Gupta, now we have to take pairs of points for
each dimension, using all of the stars. Below I only print the output
for the first parameter (first slot), where the two columns reflect all
possible combinations between pairs of points, for one star and the
other.

\begin{Shaded}
\begin{Highlighting}[]
\CommentTok{# EXTRACT PAIRS OF POINTS -----------------------------------------------------}

\NormalTok{pairs.points <-}\StringTok{ }\KeywordTok{lapply}\NormalTok{(}\DecValTok{1}\OperatorTok{:}\KeywordTok{length}\NormalTok{(params), }\ControlFlowTok{function}\NormalTok{(x) }
  \KeywordTok{lapply}\NormalTok{(}\DecValTok{1}\OperatorTok{:}\KeywordTok{ncol}\NormalTok{(out[[x]]), }\ControlFlowTok{function}\NormalTok{(y) }
    \KeywordTok{t}\NormalTok{(}\KeywordTok{combn}\NormalTok{(out[[x]][, y], }\DecValTok{2}\NormalTok{))))}

\NormalTok{pairs.points <-}\StringTok{ }\KeywordTok{lapply}\NormalTok{(pairs.points, }\ControlFlowTok{function}\NormalTok{(x) }\KeywordTok{do.call}\NormalTok{(rbind, x))}

\KeywordTok{print}\NormalTok{(pairs.points[[}\DecValTok{1}\NormalTok{]])}
\end{Highlighting}
\end{Shaded}

\begin{verbatim}
##             [,1]          [,2]
##  [1,]  0.0000000 -5.877853e-01
##  [2,]  0.0000000 -9.510565e-01
##  [3,]  0.0000000 -9.510565e-01
##  [4,]  0.0000000 -5.877853e-01
##  [5,]  0.0000000  5.877853e-01
##  [6,]  0.0000000  9.510565e-01
##  [7,]  0.0000000  9.510565e-01
##  [8,]  0.0000000  5.877853e-01
##  [9,]  0.0000000  1.224647e-16
## [10,] -0.5877853 -9.510565e-01
## [11,] -0.5877853 -9.510565e-01
## [12,] -0.5877853 -5.877853e-01
## [13,] -0.5877853  5.877853e-01
## [14,] -0.5877853  9.510565e-01
## [15,] -0.5877853  9.510565e-01
## [16,] -0.5877853  5.877853e-01
## [17,] -0.5877853  1.224647e-16
## [18,] -0.9510565 -9.510565e-01
## [19,] -0.9510565 -5.877853e-01
## [20,] -0.9510565  5.877853e-01
## [21,] -0.9510565  9.510565e-01
## [22,] -0.9510565  9.510565e-01
## [23,] -0.9510565  5.877853e-01
## [24,] -0.9510565  1.224647e-16
## [25,] -0.9510565 -5.877853e-01
## [26,] -0.9510565  5.877853e-01
## [27,] -0.9510565  9.510565e-01
## [28,] -0.9510565  9.510565e-01
## [29,] -0.9510565  5.877853e-01
## [30,] -0.9510565  1.224647e-16
## [31,] -0.5877853  5.877853e-01
## [32,] -0.5877853  9.510565e-01
## [33,] -0.5877853  9.510565e-01
## [34,] -0.5877853  5.877853e-01
## [35,] -0.5877853  1.224647e-16
## [36,]  0.5877853  9.510565e-01
## [37,]  0.5877853  9.510565e-01
## [38,]  0.5877853  5.877853e-01
## [39,]  0.5877853  1.224647e-16
## [40,]  0.9510565  9.510565e-01
## [41,]  0.9510565  5.877853e-01
## [42,]  0.9510565  1.224647e-16
## [43,]  0.9510565  5.877853e-01
## [44,]  0.9510565  1.224647e-16
## [45,]  0.5877853  1.224647e-16
## [46,]  9.0880682 -1.903335e-01
## [47,]  9.0880682 -3.734368e+00
## [48,]  9.0880682 -5.088068e+00
## [49,]  9.0880682 -3.734368e+00
## [50,]  9.0880682 -1.903335e-01
## [51,]  9.0880682  4.190334e+00
## [52,]  9.0880682  7.734368e+00
## [53,]  9.0880682  7.734368e+00
## [54,]  9.0880682  4.190334e+00
## [55,] -0.1903335 -3.734368e+00
## [56,] -0.1903335 -5.088068e+00
## [57,] -0.1903335 -3.734368e+00
## [58,] -0.1903335 -1.903335e-01
## [59,] -0.1903335  4.190334e+00
## [60,] -0.1903335  7.734368e+00
## [61,] -0.1903335  7.734368e+00
## [62,] -0.1903335  4.190334e+00
## [63,] -3.7343676 -5.088068e+00
## [64,] -3.7343676 -3.734368e+00
## [65,] -3.7343676 -1.903335e-01
## [66,] -3.7343676  4.190334e+00
## [67,] -3.7343676  7.734368e+00
## [68,] -3.7343676  7.734368e+00
## [69,] -3.7343676  4.190334e+00
## [70,] -5.0880682 -3.734368e+00
## [71,] -5.0880682 -1.903335e-01
## [72,] -5.0880682  4.190334e+00
## [73,] -5.0880682  7.734368e+00
## [74,] -5.0880682  7.734368e+00
## [75,] -5.0880682  4.190334e+00
## [76,] -3.7343676 -1.903335e-01
## [77,] -3.7343676  4.190334e+00
## [78,] -3.7343676  7.734368e+00
## [79,] -3.7343676  7.734368e+00
## [80,] -3.7343676  4.190334e+00
## [81,] -0.1903335  4.190334e+00
## [82,] -0.1903335  7.734368e+00
## [83,] -0.1903335  7.734368e+00
## [84,] -0.1903335  4.190334e+00
## [85,]  4.1903335  7.734368e+00
## [86,]  4.1903335  7.734368e+00
## [87,]  4.1903335  4.190334e+00
## [88,]  7.7343676  7.734368e+00
## [89,]  7.7343676  4.190334e+00
## [90,]  7.7343676  4.190334e+00
\end{verbatim}

Now I assume that I can compute the variogram, using the following
formula:

\begin{equation}
\gamma_i(h) = \frac{1}{2N_h} \sum_{j=1}^{N_h}(y_j^{i(h)} - y_j)^2
\end{equation}

where \(N_h\) is the number of pairs differing by a distance \(h\).

\begin{Shaded}
\begin{Highlighting}[]
\CommentTok{# COMPUTE VARIOGRAM -----------------------------------------------------------}

\CommentTok{# Define formula}
\NormalTok{variogram <-}\StringTok{ }\ControlFlowTok{function}\NormalTok{(x) }\DecValTok{1}\OperatorTok{/}\StringTok{ }\NormalTok{(}\DecValTok{2} \OperatorTok{*}\StringTok{ }\KeywordTok{nrow}\NormalTok{(x)) }\OperatorTok{*}\StringTok{ }\KeywordTok{sum}\NormalTok{(x[, }\DecValTok{1}\NormalTok{] }\OperatorTok{-}\StringTok{ }\NormalTok{x[, }\DecValTok{2}\NormalTok{]) }\OperatorTok{^}\StringTok{ }\DecValTok{2}

\CommentTok{# Compute}
\NormalTok{variogr <-}\StringTok{ }\KeywordTok{unlist}\NormalTok{(}\KeywordTok{lapply}\NormalTok{(pairs.points, variogram))}

\NormalTok{variogr}
\end{Highlighting}
\end{Shaded}

\begin{verbatim}
## [1] 86.4416043  0.1388889 60.9403561
\end{verbatim}

These three numbers reflect the variogram for the three parameters.

Now, the covariogram. According to Razavi and Gupta, it should be
computed as follows:

\begin{equation}
C(h) = \frac{1}{2} COV(y_j^{i(h)}, y_j)
\end{equation}

\begin{Shaded}
\begin{Highlighting}[]
\CommentTok{# COMPUTE COVARIOGRAM ---------------------------------------------------------}

\NormalTok{covariog <-}\StringTok{ }\KeywordTok{unlist}\NormalTok{(}\KeywordTok{lapply}\NormalTok{(pairs.points, }\ControlFlowTok{function}\NormalTok{(x) }\DecValTok{1} \OperatorTok{/}\StringTok{ }\NormalTok{(}\DecValTok{2} \OperatorTok{*}\StringTok{ }\KeywordTok{nrow}\NormalTok{(x)) }\OperatorTok{*}\StringTok{ }\KeywordTok{cov}\NormalTok{(x[, }\DecValTok{1}\NormalTok{], x[, }\DecValTok{2}\NormalTok{])))}

\NormalTok{covariog}
\end{Highlighting}
\end{Shaded}

\begin{verbatim}
## [1] 0.0003195397 0.0702550209 0.4909264736
\end{verbatim}

I am not sure if the covariogram results make much sense\ldots{}

How do I compute the total variance?

\end{document}
